\documentclass[11pt, oneside]{article}   	% use "amsart" instead of "article" for AMSLaTeX format
\usepackage{geometry}                		% See geometry.pdf to learn the layout options. There are lots.
\geometry{letterpaper}                   		% ... or a4paper or a5paper or ... 
%\geometry{landscape}                		% Activate for for rotated page geometry
%\usepackage[parfill]{parskip}    		% Activate to begin paragraphs with an empty line rather than an indent
\usepackage{graphicx}				% Use pdf, png, jpg, or eps§ with pdflatex; use eps in DVI mode
								% TeX will automatically convert eps --> pdf in pdflatex		
\usepackage{amssymb}
\usepackage{amsmath}
\usepackage{hyperref}

\title{Formulating Laplacian Matrix Generation as a N-D Stencil Problem}
\author{Nathan Zhang}
%\date{}							% Activate to display a given date or no date

\begin{document}
\maketitle
%\section{}
%\subsection{}
As per Professor James Demmel's lecture note\footnote{\url{http://www.cs.berkeley.edu/~demmel/cs267_Spr14/Lectures/lecture21_structured_jwd14_4pp.pdf}}, we can generate the heat equation matrix as $T = I - z*L$ with $z = \frac{C \Delta t}{h^2}$. $C$ is the heat diffusivity constant and $h$ as the position step size, or distance between two sample points.  $L$, the Laplacian matrix, assumes that the mesh in question is fully embedded within a larger superstructure, so that there are no true boundary positions. 

The Laplacian matrix is defined as follows for a n-D space $S$ where $p_i$ is the coordinate of point $i$ of degree $n$ \footnote{\url{http://en.wikipedia.org/wiki/Laplacian_matrix}}\\
\[
S_{p_i, p_j} =
  \begin{cases}
  	2^n & \text{if } p_i = p_j\\
	-1 & \text{if } ||p_ip_j||_1 = 1\\
	0 & \text{otherwise}
  \end{cases}
\]
We can thus express $S$ as a matrix of dimension $2n$, the first $n$ being $p_i$ and the second being $p_j$. To obtain our final 2D matrix, we can simply reshape by linearizing the first n dimensions and the last n.

\end{document}  