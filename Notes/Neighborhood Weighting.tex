\documentclass[11pt, oneside]{article}   	% use "amsart" instead of "article" for AMSLaTeX format
\usepackage{geometry}                		% See geometry.pdf to learn the layout options. There are lots.
\geometry{letterpaper}                   		% ... or a4paper or a5paper or ... 
%\geometry{landscape}                		% Activate for for rotated page geometry
%\usepackage[parfill]{parskip}    		% Activate to begin paragraphs with an empty line rather than an indent
\usepackage{graphicx}				% Use pdf, png, jpg, or eps§ with pdflatex; use eps in DVI mode
								% TeX will automatically convert eps --> pdf in pdflatex		
\usepackage{amssymb}
\usepackage{amsmath}
\usepackage{tikz}
\usepgflibrary{arrows}


\title{Neighborhood weighting in N-dim Space}
\author{Nathan Zhang}
%\date{}							% Activate to display a given date or no date

\begin{document}
\maketitle

\section{Restriction}
\subsection{Problem}
For any restriction weighting algorithm, the uniform matrix must be a fixed point value, reducing to a smaller matrix of the same values. Furthermore, the sum of weights around a point must be 1 to prevent value inflation/deflation.\\

Let $P$ be the N-dimensional coordinate of a non-boundary restriction point. For each neighbor $i$ at position $p_i$ with value $v_i$ in $P$'s neighborhood, let $\delta_i$ be the vector $p_i - P$. We then set the weight of this neighbor to be $2^{-(||\delta_i||_1 - N)}$. That is, the value of $P$ post-restriction is $$\Sigma_{i \in neighborhood(P)} 2^{-(||p_i - P||_1 - N)} * v_i$$

\subsection{Weight Calculation}
In $N$-dimensional space the cumulative weight is $\Sigma_{i \in neighborhood(P)} 2^{-(||p_i - P||_1 - N)}$. Each $\delta$ falls in $\{-1, 0, 1\}^n$. There are $\dbinom{N}{k} \cdot 2^k$ $\delta$s of 1-norm $k$, and therefore the sum of all weights at the $||\delta||_1=k$ level is $\dbinom{N}{k} \cdot 2^k \cdot 2^{-(k+N)}$ or $\dbinom{N}{k}2^{-N}$. Since $\Sigma_{k=0}^{N} \dbinom{N}{k} = 2^N$, we arrive at the cumulative weight of $2^N \cdot 2^{-N} = 1$

\section{Interpolation}
\subsection{Problem}
When interpolating in an $N$-dimensional space we need to find an algorithm to interpolate into spaces.\\

Let $P$ be a point we wish to interpolate to in this case. The value of this point is equal to 
$$
\Sigma_{n \in neighborhood(P)}
\begin{cases}
	2^{-||n-P||_1} \cdot value(n)& \text{if } defined(n)\\
	0 & \text{otherwise}
\end{cases}
$$
In order for an algorithm to be valid, the sum of total weights on this point must also be 1.

\subsection{Weight Calculation}
By construction of our interpolation target matrix, we know that all known points in the neighborhood of every point in our target matrix have the same norm. If the neighborhood of a point consists defined points with 1-norm $k$, then each point has the weight $2^{-k}$ We can prove that the sum of contributions is 1 by induction.\\

In the base case, where $N=0, k=0$, we have exactly 1 point (the point itself, so it must be defined), and has a weight of $2^0 = 1$.\\

We have two inductive cases, $N+1$-dimension, $||\delta||_1 = k+1$; $N+1$-dimension, $||\delta||_1 = 0$\\

In the first case, we can generate the $||\delta||_1 = k+1$ vectors from appending either $-1$ or $1$ onto each existing $||\delta||_1 = k$ vector, giving twice as many points, or $2^{k+1}$ points in $N+1$ dimensions.\\

In the second case, we know that there is exactly one point with $||\delta||_1 = 0$, the point itself. This has a weight of $2^0 = 1$.\\

\section{Formulation as Stencil Problems}
\subsection{Restriction}

\subsection{Interpolation}

\end{document}  